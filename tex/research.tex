% State of research
\chapter{Hauptteil}

\section{Link}
Das ist ein \hyperref[ch:intro]{Link}.

\section{Liste}
\begin{itemize}
    \item \textbf{Unzureichendes Wissen:} Studenten sind besonders zu Beginn ihres Studiums nicht mit wissenschaftlicher Methodik vertraut und kennen \ggf{} nicht die Regeln zum korrekten Zitieren oder Referenzieren.
    \item \textbf{Effizienzsteigerung:} Durch Aneignung fremder Leistungen wird der eigene Arbeitsaufwand reduziert und bessere Ergebnisse können in kürzerer Zeit erzielt werden.
    \item \textbf{Einstellung zur Aufgabe:} Vereinzelt fehlt das Verständnis für die Bedeutsamkeit der Aufgabenstellung. Dies kann in manchen Fällen auf ein fehlendes Vertrauen in die Lehrperson zurückgeführt werden.
\end{itemize}

\section{Aufzählung}
\begin{enumerate}[align=left, leftmargin=3em]
    \item [{\bfseries RQ1.}] Was wird in der Wissenschaft als Plagiat bezeichnet?
    \item [{\bfseries RQ2.}] Wie können Plagiate vermieden, erkannt und geahndet werden?
    \item [{\bfseries RQ3.}] Welche frei verfügbaren Tools eignen sich für die CaPD?
    \item [{\bfseries RQ4.}] Wo besteht noch Verbesserungspotential?
\end{enumerate}

\section{Tabelle}
\cref{tab:coverage} skizziert seine wesentlichen Ideen.
% Please add the following required packages to your document preamble:
% \usepackage{booktabs}
% \usepackage{multirow}
% \usepackage{graphicx}
% \usepackage[table,xcdraw]{xcolor}
% If you use beamer only pass "xcolor=table" option, i.e. \documentclass[xcolor=table]{beamer}
% \usepackage[normalem]{ulem}
% \useunder{\uline}{\ul}{}
% \usepackage{lscape}
\begin{landscape}
  \begin{table}
  \centering
  \resizebox{\textheight}{!}{%
  \begin{tabular}{@{}
  >{\columncolor[HTML]{C0C0C0}}l |l|l|l|l|l|l|ll@{}}
  \cellcolor[HTML]{9B9B9B}{\color[HTML]{9B9B9B} } &
    \multicolumn{2}{c|}{\cellcolor[HTML]{C0C0C0}{\ul \textbf{Lexikalische Erkennung}}} &
    \multicolumn{2}{c|}{\cellcolor[HTML]{C0C0C0}{\ul \textbf{Syntaktische Erkennung}}} &
    \multicolumn{2}{c|}{\cellcolor[HTML]{C0C0C0}{\ul \textbf{Semantische Erkennung}}} &
    \multicolumn{2}{c}{\cellcolor[HTML]{C0C0C0}{\ul \textbf{Gesamt}}} \\
  \multirow{-2}{*}{\cellcolor[HTML]{9B9B9B}{\color[HTML]{9B9B9B} }} &
    \multicolumn{1}{c|}{\cellcolor[HTML]{C0C0C0}\textbf{\(\bm{cov(S_{lex})}\)}} &
    \multicolumn{1}{c|}{\cellcolor[HTML]{C0C0C0}\textbf{Punkte}} &
    \multicolumn{1}{c|}{\cellcolor[HTML]{C0C0C0}\textbf{\(\bm{cov(S_{syn})}\)}} &
    \multicolumn{1}{c|}{\cellcolor[HTML]{C0C0C0}\textbf{Punkte}} &
    \multicolumn{1}{c|}{\cellcolor[HTML]{C0C0C0}\textbf{\(\bm{cov(S_{sem})}\)}} &
    \multicolumn{1}{c|}{\cellcolor[HTML]{C0C0C0}\textbf{Punkte}} &
    \multicolumn{1}{c|}{\cellcolor[HTML]{C0C0C0}\textbf{\(\bm{cov(S_{ges})}\)}} &
    \multicolumn{1}{c}{\cellcolor[HTML]{C0C0C0}\textbf{Punkte}} \\ \midrule
  \textbf{Ferret} &
    \(\frac{140}{154}\approx 91\% \) &
    3 &
    \(\frac{120}{164}\approx 73\%\) &
    2 &
    \(\frac{61}{169}\approx 36\%\) &
    1 &
    \multicolumn{1}{l|}{\(\frac{321}{487}\approx 66\%\)} &
    2 \\ \midrule
  \textbf{JPlag} &
    \(\frac{116}{154}\approx 75\%\) &
    2 &
    \(\frac{104}{164}\approx 63\%\) &
    2 &
    \(\frac{31}{169}\approx 18\%\) &
    0 &
    \multicolumn{1}{l|}{\(\frac{251}{487}\approx 52\%\)} &
    2 \\ \midrule
  \textbf{plagiarism\_detection} &
    n/a &
    n/a &
    n/a &
    n/a &
    n/a &
    n/a &
    \multicolumn{1}{l|}{n/a} &
    n/a \\ \midrule
  \textbf{PlagZap} &
    \(\frac{95}{154}\approx 62\%\) &
    2 &
    \(\frac{49}{164}\approx 30\%\) &
    1 &
    \(\frac{25}{169}\approx 15\%\) &
    0 &
    \multicolumn{1}{l|}{\(\frac{169}{487}\approx 35\%\)} &
    1 \\ \midrule
  \textbf{\begin{tabular}[c]{@{}l@{}}Sanchez-Perez \etal{}\\ (PAN 2015)\end{tabular}} &
    \(\frac{154}{154}=100\%\) &
    3 &
    \(\frac{158}{164}\approx 96\%\) &
    3 &
    \(\frac{111}{169}\approx 66\%\) &
    2 &
    \multicolumn{1}{l|}{\(\frac{423}{487}\approx 87\%\)} &
    3 \\ \midrule
  \textbf{Sherlock} &
    \(\frac{117}{154}\approx 76\%\) &
    3 &
    \(\frac{158}{164}\approx 96\%\) &
    3 &
    \(\frac{114}{169}\approx 67\%\) &
    2 &
    \multicolumn{1}{l|}{\(\frac{389}{487}\approx 80\%\)} &
    3 \\ \midrule
  \textbf{SIM} &
    \(\frac{55}{154}\approx 36\%\) &
    1 &
    \(\frac{31}{164}\approx 19\%\) &
    0 &
    \(\frac{29}{169}\approx 17\%\) &
    0 &
    \multicolumn{1}{l|}{\(\frac{115}{487}\approx 24\%\)} &
    0 \\ \midrule
  \textbf{text-matcher} &
    \(\frac{116}{154}\approx 75\%\) &
    2 &
    \(\frac{77}{164}\approx 47\%\) &
    1 &
    \(\frac{12}{169}\approx 7\%\) &
    0 &
    \multicolumn{1}{l|}{\(\frac{205}{487}\approx 42\%\)} &
    1 \\ \midrule
  \textbf{WCopyfind} &
    \(\frac{135}{154}\approx 88\%\) &
    3 &
    \(\frac{110}{164}\approx 67\%\) &
    2 &
    \(\frac{46}{169}\approx 27\%\) &
    1 &
    \multicolumn{1}{l|}{\(\frac{291}{487}\approx 60\%\)} &
    2
  \end{tabular}%
  }
  \caption{Vergleich der wortbasierten Plagiatsfallüberdeckung und Einordnung auf Punkteskala}\label{tab:coverage}
  \end{table}
  \end{landscape}

\section{Bild}
\begin{figure}
    \centering
    \includegraphics[width=0.9\textwidth]{sherlock-overview}
    \caption{Ergebnisübersicht von \emph{Sherlock}}\label{fig:sherlockSum}
\end{figure}

\section{Tikz}
\begin{figure}
    \centering
    \resizebox{0.9\textwidth}{!}{%
        \begin{tikzpicture}[grow=right, level distance=4cm, sibling distance=0.5cm]
            \Tree [.Plagiat
                    [.{andere Plagiate} {Plagiat mit Erlaubnis} Selbstplagiat Ghostwriting ]
                    [.intelligent
                            [.ideenbasiert Struktur\-plagiat ]
                            [.semantisch {Übersetzungs\-plagiat} {unzulässiges Paraphrasieren} ]
                    ]
                    [.{buchstäblich/\\wörtlich}
                            [.syntaktisch {Synonym\-substitution} {technische Verschleierung} ]
                            [.lexikalisch {schein\-paraphrasieren} {wörtliches Kopieren} ]
                    ]
            ]
        \end{tikzpicture}
    }%
    \caption{Plagiat -- Eine Taxonomie}\label{fig:taxonomy}% chktex 8
\end{figure}

\section{Textzitat}
So definiert der Duden Plagiat als \textcquote{Dudenredaktion.2018}{unrechtmäßige Aneignung von Gedanken, Ideen \oae{} eines anderen auf künstlerischem oder wissenschaftlichem Gebiet und ihre Veröffentlichung; Diebstahl geistigen Eigentums}.

\section{Blockzitat}
\begin{displaycquote}[1]{Manning.2018cop.2008}
    Information retrieval (IR) is finding material (usually documents) of an unstructured nature (usually text) that satisfies an information need from within large collections (usually stored on computers).
\end{displaycquote}

\section{Indirektes Zitat}
Außerdem entsteht -- insbesondere bei studentischen Arbeiten -- zumeist kein messbarer Verlust für den Autor, während dies bei einem Diebstahl sehr wohl der Fall wäre \indcite[2\psq]{TeddiFishman.2009}.

\section{Gleichung}
\begin{equation}\label{eqn:prec}
    \mbox{prec}=\frac{|\{\mbox{relevant results}\}|}{|\{\mbox{results}\}|}
\end{equation}

\section{Listing}
\begin{listing}
    \inputminted{bat}{lst/ferret.txt}
    \caption{Konsolenausgabe von \emph{Ferret}}\label{lst:ferret}
\end{listing}

\section{Langes Listing}
\begin{longlisting}
    \inputminted{xml}{lst/ferret.xml}
    \caption{Detaillierter Bericht von \emph{Ferret}}\label{lst:ferretXml}
\end{longlisting}